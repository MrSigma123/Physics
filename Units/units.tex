\documentclass{article}
\usepackage{amsmath}
\usepackage{siunitx}
\usepackage{booktabs}
\usepackage{hyperref}
\usepackage{graphicx}
\title{Units Transformation in Physics}
\author{}
\date{}

\begin{document}

\maketitle

\section*{Introduction}
In physics, proper unit conversion is crucial for solving problems and ensuring consistency in calculations. This document provides a clear guide to transforming units, starting with basic examples and progressing to more advanced ones.

\section{Basic Unit Transformations}

\subsection{Length}
Converting between common units of length:
\begin{align*}
1 \text{ m} &= 100 \text{ cm}, \\
1 \text{ km} &= 1000 \text{ m}, \\
1 \text{ inch} &= 2.54 \text{ cm}.
\end{align*}

Example:
Convert \SI{5}{km} to meters:
\begin{align*}
\SI{5}{km} \times \frac{1000 \text{ m}}{1 \text{ km}} = \SI{5000}{m}.
\end{align*}

\subsection{Time}
Common conversions for time:
\begin{align*}
1 \text{ minute} &= 60 \text{ seconds}, \\
1 \text{ hour} &= 3600 \text{ seconds}.
\end{align*}

Example:
Convert \SI{2.5}{hours} to seconds:
\begin{align*}
\SI{2.5}{hours} \times \frac{3600 \text{ s}}{1 \text{ hour}} = \SI{9000}{s}.
\end{align*}

\section{Intermediate Unit Transformations}

\subsection{Speed}
Converting speed from kilometers per hour to meters per second:
\begin{align*}
1 \text{ km/h} &= \frac{1000 \text{ m}}{3600 \text{ s}} = \SI{0.2778}{m/s}.
\end{align*}

Example:
Convert \SI{90}{km/h} to \si{m/s}:
\begin{align*}
\SI{90}{km/h} \times \frac{1000 \text{ m}}{3600 \text{ s}} = \SI{25}{m/s}.
\end{align*}

\subsection{Force}
Given $F = ma$, where $m$ is in kilograms and $a$ is in meters per second squared, the unit of force is derived as:
\begin{align*}
[F] &= \text{kg} \cdot \text{m/s}^2 = \text{N} \quad (\text{Newton}).
\end{align*}

\section{Advanced Unit Transformations}

\subsection{Energy}
Energy in the form of kinetic energy is given by $E = \frac{1}{2}mv^2$. Suppose $m$ is given in grams and $v$ in kilometers per hour, and we want the result in joules:

Given:
\begin{align*}
m &= \SI{500}{g} = \SI{0.5}{kg}, \\
v &= \SI{72}{km/h} = \SI{20}{m/s}.
\end{align*}

Substitute into the equation:
\begin{align*}
E &= \frac{1}{2} \cdot \SI{0.5}{kg} \cdot (\SI{20}{m/s})^2 = \SI{100}{J}.
\end{align*}

\subsection{Pressure}
Pressure is defined as $P = \frac{F}{A}$. Suppose we have force in newtons and area in square centimeters, and we need pressure in pascals:

Given:
\begin{align*}
F &= \SI{50}{N}, \\
A &= \SI{20}{cm^2} = \SI{0.002}{m^2}.
\end{align*}

Substitute into the equation:
\begin{align*}
P &= \frac{\SI{50}{N}}{\SI{0.002}{m^2}} = \SI{25000}{Pa}.
\end{align*}

\section*{Conclusion}
Understanding and correctly applying unit transformations is essential for solving a wide range of problems in physics. By mastering both basic and advanced conversions, one can ensure accuracy and consistency in calculations.

\end{document}

